\documentclass[10pt,a4paper]{scrreprt}
\usepackage[utf8]{inputenc}
\usepackage{amsmath}
\usepackage{tipa}
\usepackage{amsfonts}
\usepackage{german}
\usepackage{graphicx}
\usepackage{amssymb}
\usepackage{parskip}
\usepackage{german,longtable}
\usepackage[export]{adjustbox}
\usepackage{microtype}
\usepackage[hidelinks]{hyperref}

\hyphenation{Er-folgs-wahr-schein-lich-keit Klartext-alphabet Kombina-tions-feld IControl-Transpo-Encryption IControl-Cube-Attack Initia-lisie-rungs-vektor Block-verkettungs-modus Ausgabe-daten-strom Eingabe-daten-strom}

\DisableLigatures{}
\setcounter{tocdepth}{4}
\setcounter{secnumdepth}{4}

\makeatletter \setlength\@fptop{0\p@}
\makeatother

\title{Dokumentation zu den Komponenten in CrypTool v2 (CT2)}
\author{Das CrypTool 2-Team}

\newcommand{\HRule}{\rule{\linewidth}{0.5mm}}

\begin{document}

\begin{titlepage}
\begin{center}
\hspace{0pt}\\[2.5cm]

\HRule \\[0.4cm]
{ \huge \bfseries Dokumentation zu den Komponenten in CrypTool v2 (CT2) }\\[0.4cm]
\HRule \\[1.5cm]

\begin{minipage}{0.4\textwidth}
\begin{flushleft} \large
\emph{Autoren:} \\
Das CrypTool 2 Team
\end{flushleft}
\end{minipage}
\begin{minipage}{0.4\textwidth}
\begin{flushright} \large
\emph{Version:} \\
$VERSION$
\end{flushright}
\end{minipage}
\vfill
{\large \today}
\end{center}
\newpage
\large
Dieses Dokument gibt eine Übersicht über die in {\bf CrypTool v2} enthaltenen Komponenten.
\newpage
\end{titlepage}

\tableofcontents
\newpage

$CONTENT$

\end{document}
